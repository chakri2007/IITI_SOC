
\documentclass[12pt]{article}
\usepackage[a4paper, margin=1in]{geometry}
\usepackage{hyperref}
\usepackage{enumitem}
\usepackage{titlesec}
\usepackage{setspace}
\usepackage{fancyhdr}
\setstretch{1.15}
\pagestyle{fancy}
\fancyhf{}
\rhead{IITISOC 2025 Proposal}
\lhead{Dual-Drone Framework}
\rfoot{\thepage}
\titleformat{\section}{\normalfont\Large\bfseries}{}{0pt}{}
\titleformat{\subsection}{\normalfont\large\bfseries}{}{0pt}{}

\title{IITISOC 2025\\\large Coordinated Dual-Drone Framework for Coverage Based Irrigation Tasks}
\author{}
\date{}

\begin{document}
\maketitle

\section*{Team Details}
\begin{itemize}[leftmargin=*]
  \item \textbf{Team Leader:} Burra Venkata Chakrapani (240021005)\\
  GitHub: \texttt{chakri2007}, Discord: chakrapani2006, \href{https://www.linkedin.com/in/chakrapani-burra-086878352}{LinkedIn}\\
  Skills: ROS2, PX4, Python
  \item \textbf{Team Member 2:} Bhav Makhija (240021004)\\
  GitHub: \texttt{Bhavdex}, Discord: Bhavj, \href{https://www.linkedin.com/in/bhav-makhija}{LinkedIn}\\
  Skills: Algorithm Development
  \item \textbf{Team Member 3:} Sathvik (240021001)\\
  GitHub: \texttt{Sathvik557}, Discord: Sathvik0586, \href{https://www.linkedin.com/in/amadala-sathvik-333b7a323}{LinkedIn}\\
  Skills: PX4, CAD
  \item \textbf{Team Member 4:} Srujana Patil (240021018)\\
  GitHub: \texttt{Sruj-08}, Discord: Sruj08, \href{https://www.linkedin.com/in/srujana-patil-620677324}{LinkedIn}\\
  Skills: Python
  \item \textbf{Team Member 5:} Sana Tejasri Lakshmi (240041033)\\
  GitHub: \texttt{Tejasri676}, Discord: Sana676, \href{https://www.linkedin.com/in/tejasri-lakshmi-sana-b6951b321}{LinkedIn}\\
  Skills: Python
\end{itemize}

\section*{Domain and Problem Statement}
\textbf{Domain:} Intelligent Vehicles and Robotics\\
\textbf{Problem Statement:} Coordinated Dual-Drone Framework for Coverage Based Irrigation Tasks\\
\textbf{PS Number:} 04 \\
\textbf{Preference Number:} 01

\section*{Project Solution}
We describe our plan to approach the project step-by-step including geotagging, drone communication, and energy management to optimize irrigation.

To minimize energy, surveillance drones start with an empty water tank. They only switch roles to irrigation if needed.

\begin{center}
\begin{tabular}{|l|c|c|c|}
\hline
\textbf{Drone Condition} & \textbf{Speed (m/s)} & \textbf{Flight Time} & \textbf{Recharge Time} \\
\hline
With Water & 4 & 15 min & 25 min \\
Without Water & 8 & 30 min & 25 min \\
\hline
\end{tabular}
\end{center}

Drones start from the farthest point and move inward for better efficiency.

\section*{Launch and Path Planning}
Using QGroundControl (QGC), we define geofences and generate a lawnmower path. Waypoints are exported as a KML file and include an `IsVisited` flag.

\section*{Geotagging and Severity Level}
Waypoints are placed 25m apart using latitude and longitude increments. Each is assigned a severity level from 1 (low) to 5 (high). Levels 2–5 trigger irrigation with amount based on severity.

\section*{Battery Recharging Logic}
Drones track battery and task load to ensure one drone is always operational.

\subsection*{Battery Thresholds for Return-to-Base}
Drones return if the energy is only enough for return using the $x'$ method:
\begin{itemize}
  \item $X$ = max travel range on full battery
  \item $Y$ = current distance to charging station
  \item $X' = X - Y$
\end{itemize}
If next waypoint distance > $X'$, drone returns to base.

\section*{Swarm Coordination}
Drones share GPS location, battery, water levels and progress to avoid duplication, support role switching, and balance load.

\section*{Collision Avoidance}
Drones fly at different altitudes. If on the same level, a PX4 obstacle avoidance system with a depth camera adjusts flight path during AUTO missions.

\section*{Time Efficiency}
Launching from the farthest point helps reduce return trip energy use. Parallel operation and task sharing increases speed and reduces mission time.

\section*{Further Improvements}
We plan to run MATLAB simulations to optimize our strategy before real-world deployment.

\section*{Proposal Schedule}

\begin{itemize}
\item \textbf{Week 1:} Mission planning with QGC, generate lawnmower path and waypoint JSON.
\item \textbf{Week 2:} Configure surveillance drone for geotagging and severity assignment.
\item \textbf{Week 3:} Irrigation drone reads geotags and performs targeted irrigation.
\item \textbf{Week 4:} Implement real-time battery monitoring and auto return-to-base logic.
\item \textbf{Week 5:} Add swarm coordination for dynamic workload sharing.
\item \textbf{Week 6:} Altitude separation and PX4 obstacle avoidance; MATLAB simulations.
\item \textbf{Week 7:} Automate launch files, setup GitHub repo with README.
\item \textbf{Week 8:} Optimize code, finalize report and documentation.
\end{itemize}

\end{document}
